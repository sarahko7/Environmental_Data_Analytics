\documentclass[]{article}
\usepackage{lmodern}
\usepackage{amssymb,amsmath}
\usepackage{ifxetex,ifluatex}
\usepackage{fixltx2e} % provides \textsubscript
\ifnum 0\ifxetex 1\fi\ifluatex 1\fi=0 % if pdftex
  \usepackage[T1]{fontenc}
  \usepackage[utf8]{inputenc}
\else % if luatex or xelatex
  \ifxetex
    \usepackage{mathspec}
  \else
    \usepackage{fontspec}
  \fi
  \defaultfontfeatures{Ligatures=TeX,Scale=MatchLowercase}
\fi
% use upquote if available, for straight quotes in verbatim environments
\IfFileExists{upquote.sty}{\usepackage{upquote}}{}
% use microtype if available
\IfFileExists{microtype.sty}{%
\usepackage{microtype}
\UseMicrotypeSet[protrusion]{basicmath} % disable protrusion for tt fonts
}{}
\usepackage[margin=2.54cm]{geometry}
\usepackage{hyperref}
\hypersetup{unicode=true,
            pdftitle={Assignment 4: Data Wrangling},
            pdfauthor={Sarah Ko},
            pdfborder={0 0 0},
            breaklinks=true}
\urlstyle{same}  % don't use monospace font for urls
\usepackage{color}
\usepackage{fancyvrb}
\newcommand{\VerbBar}{|}
\newcommand{\VERB}{\Verb[commandchars=\\\{\}]}
\DefineVerbatimEnvironment{Highlighting}{Verbatim}{commandchars=\\\{\}}
% Add ',fontsize=\small' for more characters per line
\usepackage{framed}
\definecolor{shadecolor}{RGB}{248,248,248}
\newenvironment{Shaded}{\begin{snugshade}}{\end{snugshade}}
\newcommand{\KeywordTok}[1]{\textcolor[rgb]{0.13,0.29,0.53}{\textbf{#1}}}
\newcommand{\DataTypeTok}[1]{\textcolor[rgb]{0.13,0.29,0.53}{#1}}
\newcommand{\DecValTok}[1]{\textcolor[rgb]{0.00,0.00,0.81}{#1}}
\newcommand{\BaseNTok}[1]{\textcolor[rgb]{0.00,0.00,0.81}{#1}}
\newcommand{\FloatTok}[1]{\textcolor[rgb]{0.00,0.00,0.81}{#1}}
\newcommand{\ConstantTok}[1]{\textcolor[rgb]{0.00,0.00,0.00}{#1}}
\newcommand{\CharTok}[1]{\textcolor[rgb]{0.31,0.60,0.02}{#1}}
\newcommand{\SpecialCharTok}[1]{\textcolor[rgb]{0.00,0.00,0.00}{#1}}
\newcommand{\StringTok}[1]{\textcolor[rgb]{0.31,0.60,0.02}{#1}}
\newcommand{\VerbatimStringTok}[1]{\textcolor[rgb]{0.31,0.60,0.02}{#1}}
\newcommand{\SpecialStringTok}[1]{\textcolor[rgb]{0.31,0.60,0.02}{#1}}
\newcommand{\ImportTok}[1]{#1}
\newcommand{\CommentTok}[1]{\textcolor[rgb]{0.56,0.35,0.01}{\textit{#1}}}
\newcommand{\DocumentationTok}[1]{\textcolor[rgb]{0.56,0.35,0.01}{\textbf{\textit{#1}}}}
\newcommand{\AnnotationTok}[1]{\textcolor[rgb]{0.56,0.35,0.01}{\textbf{\textit{#1}}}}
\newcommand{\CommentVarTok}[1]{\textcolor[rgb]{0.56,0.35,0.01}{\textbf{\textit{#1}}}}
\newcommand{\OtherTok}[1]{\textcolor[rgb]{0.56,0.35,0.01}{#1}}
\newcommand{\FunctionTok}[1]{\textcolor[rgb]{0.00,0.00,0.00}{#1}}
\newcommand{\VariableTok}[1]{\textcolor[rgb]{0.00,0.00,0.00}{#1}}
\newcommand{\ControlFlowTok}[1]{\textcolor[rgb]{0.13,0.29,0.53}{\textbf{#1}}}
\newcommand{\OperatorTok}[1]{\textcolor[rgb]{0.81,0.36,0.00}{\textbf{#1}}}
\newcommand{\BuiltInTok}[1]{#1}
\newcommand{\ExtensionTok}[1]{#1}
\newcommand{\PreprocessorTok}[1]{\textcolor[rgb]{0.56,0.35,0.01}{\textit{#1}}}
\newcommand{\AttributeTok}[1]{\textcolor[rgb]{0.77,0.63,0.00}{#1}}
\newcommand{\RegionMarkerTok}[1]{#1}
\newcommand{\InformationTok}[1]{\textcolor[rgb]{0.56,0.35,0.01}{\textbf{\textit{#1}}}}
\newcommand{\WarningTok}[1]{\textcolor[rgb]{0.56,0.35,0.01}{\textbf{\textit{#1}}}}
\newcommand{\AlertTok}[1]{\textcolor[rgb]{0.94,0.16,0.16}{#1}}
\newcommand{\ErrorTok}[1]{\textcolor[rgb]{0.64,0.00,0.00}{\textbf{#1}}}
\newcommand{\NormalTok}[1]{#1}
\usepackage{graphicx,grffile}
\makeatletter
\def\maxwidth{\ifdim\Gin@nat@width>\linewidth\linewidth\else\Gin@nat@width\fi}
\def\maxheight{\ifdim\Gin@nat@height>\textheight\textheight\else\Gin@nat@height\fi}
\makeatother
% Scale images if necessary, so that they will not overflow the page
% margins by default, and it is still possible to overwrite the defaults
% using explicit options in \includegraphics[width, height, ...]{}
\setkeys{Gin}{width=\maxwidth,height=\maxheight,keepaspectratio}
\IfFileExists{parskip.sty}{%
\usepackage{parskip}
}{% else
\setlength{\parindent}{0pt}
\setlength{\parskip}{6pt plus 2pt minus 1pt}
}
\setlength{\emergencystretch}{3em}  % prevent overfull lines
\providecommand{\tightlist}{%
  \setlength{\itemsep}{0pt}\setlength{\parskip}{0pt}}
\setcounter{secnumdepth}{0}
% Redefines (sub)paragraphs to behave more like sections
\ifx\paragraph\undefined\else
\let\oldparagraph\paragraph
\renewcommand{\paragraph}[1]{\oldparagraph{#1}\mbox{}}
\fi
\ifx\subparagraph\undefined\else
\let\oldsubparagraph\subparagraph
\renewcommand{\subparagraph}[1]{\oldsubparagraph{#1}\mbox{}}
\fi

%%% Use protect on footnotes to avoid problems with footnotes in titles
\let\rmarkdownfootnote\footnote%
\def\footnote{\protect\rmarkdownfootnote}

%%% Change title format to be more compact
\usepackage{titling}

% Create subtitle command for use in maketitle
\newcommand{\subtitle}[1]{
  \posttitle{
    \begin{center}\large#1\end{center}
    }
}

\setlength{\droptitle}{-2em}

  \title{Assignment 4: Data Wrangling}
    \pretitle{\vspace{\droptitle}\centering\huge}
  \posttitle{\par}
    \author{Sarah Ko}
    \preauthor{\centering\large\emph}
  \postauthor{\par}
    \date{}
    \predate{}\postdate{}
  

\begin{document}
\maketitle

\subsection{OVERVIEW}\label{overview}

This exercise accompanies the lessons in Environmental Data Analytics
(ENV872L) on data wrangling.

\subsection{Directions}\label{directions}

\begin{enumerate}
\def\labelenumi{\arabic{enumi}.}
\tightlist
\item
  Change ``Student Name'' on line 3 (above) with your name.
\item
  Use the lesson as a guide. It contains code that can be modified to
  complete the assignment.
\item
  Work through the steps, \textbf{creating code and output} that fulfill
  each instruction.
\item
  Be sure to \textbf{answer the questions} in this assignment document.
  Space for your answers is provided in this document and is indicated
  by the ``\textgreater{}'' character. If you need a second paragraph be
  sure to start the first line with ``\textgreater{}''. You should
  notice that the answer is highlighted in green by RStudio.
\item
  When you have completed the assignment, \textbf{Knit} the text and
  code into a single PDF file. You will need to have the correct
  software installed to do this (see Software Installation Guide) Press
  the \texttt{Knit} button in the RStudio scripting panel. This will
  save the PDF output in your Assignments folder.
\item
  After Knitting, please submit the completed exercise (PDF file) to the
  dropbox in Sakai. Please add your last name into the file name (e.g.,
  ``Salk\_A04\_DataWrangling.pdf'') prior to submission.
\end{enumerate}

The completed exercise is due on Thursday, 7 February, 2019 before class
begins.

\subsection{Set up your session}\label{set-up-your-session}

\begin{enumerate}
\def\labelenumi{\arabic{enumi}.}
\item
  Check your working directory, load the \texttt{tidyverse} package, and
  upload all four raw data files associated with the EPA Air dataset.
  See the README file for the EPA air datasets for more information
  (especially if you have not worked with air quality data previously).
\item
  Generate a few lines of code to get to know your datasets (basic data
  summaries, etc.).
\end{enumerate}

\begin{Shaded}
\begin{Highlighting}[]
\CommentTok{#1}

\CommentTok{# get working directory}
\KeywordTok{getwd}\NormalTok{()}
\end{Highlighting}
\end{Shaded}

\begin{verbatim}
## [1] "C:/Users/Sarah/Documents/Duke/Year 2/Spring 2019/Data Analytics/Environmental_Data_Analytics"
\end{verbatim}

\begin{Shaded}
\begin{Highlighting}[]
\CommentTok{# set wd to the filepath of Environmental_Data_Analytics to use relative filepath}

\CommentTok{# Load necessary package 'tidyverse'}
\KeywordTok{library}\NormalTok{(tidyverse)}
\end{Highlighting}
\end{Shaded}

\begin{verbatim}
## Warning: package 'tidyverse' was built under R version 3.5.2
\end{verbatim}

\begin{verbatim}
## -- Attaching packages ------------------------------------------------------------- tidyverse 1.2.1 --
\end{verbatim}

\begin{verbatim}
## v ggplot2 3.1.0     v purrr   0.3.0
## v tibble  2.0.1     v dplyr   0.7.8
## v tidyr   0.8.2     v stringr 1.3.1
## v readr   1.3.1     v forcats 0.3.0
\end{verbatim}

\begin{verbatim}
## Warning: package 'ggplot2' was built under R version 3.5.2
\end{verbatim}

\begin{verbatim}
## Warning: package 'tibble' was built under R version 3.5.2
\end{verbatim}

\begin{verbatim}
## Warning: package 'tidyr' was built under R version 3.5.2
\end{verbatim}

\begin{verbatim}
## Warning: package 'readr' was built under R version 3.5.2
\end{verbatim}

\begin{verbatim}
## Warning: package 'purrr' was built under R version 3.5.2
\end{verbatim}

\begin{verbatim}
## Warning: package 'dplyr' was built under R version 3.5.2
\end{verbatim}

\begin{verbatim}
## -- Conflicts ---------------------------------------------------------------- tidyverse_conflicts() --
## x dplyr::filter() masks stats::filter()
## x dplyr::lag()    masks stats::lag()
\end{verbatim}

\begin{Shaded}
\begin{Highlighting}[]
\CommentTok{# upload 4 raw data files associated w the EPA Air dataset}

\NormalTok{O3_NC2017 <-}\StringTok{ }\KeywordTok{read.csv}\NormalTok{(}\StringTok{"./Data/Raw/EPAair_O3_NC2017_raw.csv"}\NormalTok{)}
\NormalTok{O3_NC2018 <-}\StringTok{ }\KeywordTok{read.csv}\NormalTok{(}\StringTok{"./Data/Raw/EPAair_O3_NC2018_raw.csv"}\NormalTok{)}
\NormalTok{PM25_NC2017 <-}\StringTok{ }\KeywordTok{read.csv}\NormalTok{(}\StringTok{"./Data/Raw/EPAair_PM25_NC2017_raw.csv"}\NormalTok{)}
\NormalTok{PM25_NC2018 <-}\StringTok{ }\KeywordTok{read.csv}\NormalTok{(}\StringTok{"./Data/Raw/EPAair_PM25_NC2018_raw.csv"}\NormalTok{)}

\CommentTok{#2}

\CommentTok{# explore O3_NC2017}
\KeywordTok{dim}\NormalTok{(O3_NC2017)}
\end{Highlighting}
\end{Shaded}

\begin{verbatim}
## [1] 10219    20
\end{verbatim}

\begin{Shaded}
\begin{Highlighting}[]
\KeywordTok{colnames}\NormalTok{(O3_NC2017)}
\end{Highlighting}
\end{Shaded}

\begin{verbatim}
##  [1] "Date"                                
##  [2] "Source"                              
##  [3] "Site.ID"                             
##  [4] "POC"                                 
##  [5] "Daily.Max.8.hour.Ozone.Concentration"
##  [6] "UNITS"                               
##  [7] "DAILY_AQI_VALUE"                     
##  [8] "Site.Name"                           
##  [9] "DAILY_OBS_COUNT"                     
## [10] "PERCENT_COMPLETE"                    
## [11] "AQS_PARAMETER_CODE"                  
## [12] "AQS_PARAMETER_DESC"                  
## [13] "CBSA_CODE"                           
## [14] "CBSA_NAME"                           
## [15] "STATE_CODE"                          
## [16] "STATE"                               
## [17] "COUNTY_CODE"                         
## [18] "COUNTY"                              
## [19] "SITE_LATITUDE"                       
## [20] "SITE_LONGITUDE"
\end{verbatim}

\begin{Shaded}
\begin{Highlighting}[]
\KeywordTok{class}\NormalTok{(O3_NC2017)}
\end{Highlighting}
\end{Shaded}

\begin{verbatim}
## [1] "data.frame"
\end{verbatim}

\begin{Shaded}
\begin{Highlighting}[]
\KeywordTok{head}\NormalTok{(O3_NC2017)}
\end{Highlighting}
\end{Shaded}

\begin{verbatim}
##     Date Source   Site.ID POC Daily.Max.8.hour.Ozone.Concentration UNITS
## 1 3/1/17    AQS 370030005   1                                0.041   ppm
## 2 3/2/17    AQS 370030005   1                                0.046   ppm
## 3 3/3/17    AQS 370030005   1                                0.046   ppm
## 4 3/4/17    AQS 370030005   1                                0.046   ppm
## 5 3/5/17    AQS 370030005   1                                0.046   ppm
## 6 3/6/17    AQS 370030005   1                                0.048   ppm
##   DAILY_AQI_VALUE             Site.Name DAILY_OBS_COUNT PERCENT_COMPLETE
## 1              38 Taylorsville Liledoun              17              100
## 2              43 Taylorsville Liledoun              17              100
## 3              43 Taylorsville Liledoun              17              100
## 4              43 Taylorsville Liledoun              17              100
## 5              43 Taylorsville Liledoun              17              100
## 6              44 Taylorsville Liledoun              17              100
##   AQS_PARAMETER_CODE AQS_PARAMETER_DESC CBSA_CODE
## 1              44201              Ozone     25860
## 2              44201              Ozone     25860
## 3              44201              Ozone     25860
## 4              44201              Ozone     25860
## 5              44201              Ozone     25860
## 6              44201              Ozone     25860
##                      CBSA_NAME STATE_CODE          STATE COUNTY_CODE
## 1 Hickory-Lenoir-Morganton, NC         37 North Carolina           3
## 2 Hickory-Lenoir-Morganton, NC         37 North Carolina           3
## 3 Hickory-Lenoir-Morganton, NC         37 North Carolina           3
## 4 Hickory-Lenoir-Morganton, NC         37 North Carolina           3
## 5 Hickory-Lenoir-Morganton, NC         37 North Carolina           3
## 6 Hickory-Lenoir-Morganton, NC         37 North Carolina           3
##      COUNTY SITE_LATITUDE SITE_LONGITUDE
## 1 Alexander       35.9138        -81.191
## 2 Alexander       35.9138        -81.191
## 3 Alexander       35.9138        -81.191
## 4 Alexander       35.9138        -81.191
## 5 Alexander       35.9138        -81.191
## 6 Alexander       35.9138        -81.191
\end{verbatim}

\begin{Shaded}
\begin{Highlighting}[]
\KeywordTok{tail}\NormalTok{(O3_NC2017)}
\end{Highlighting}
\end{Shaded}

\begin{verbatim}
##           Date Source   Site.ID POC Daily.Max.8.hour.Ozone.Concentration
## 10214 10/25/17    AQS 371990004   1                                0.038
## 10215 10/26/17    AQS 371990004   1                                0.044
## 10216 10/27/17    AQS 371990004   1                                0.044
## 10217 10/28/17    AQS 371990004   1                                0.042
## 10218 10/30/17    AQS 371990004   1                                0.047
## 10219 10/31/17    AQS 371990004   1                                0.047
##       UNITS DAILY_AQI_VALUE    Site.Name DAILY_OBS_COUNT PERCENT_COMPLETE
## 10214   ppm              35 Mt. Mitchell              17              100
## 10215   ppm              41 Mt. Mitchell              17              100
## 10216   ppm              41 Mt. Mitchell              17              100
## 10217   ppm              39 Mt. Mitchell              17              100
## 10218   ppm              44 Mt. Mitchell              13               76
## 10219   ppm              44 Mt. Mitchell              17              100
##       AQS_PARAMETER_CODE AQS_PARAMETER_DESC CBSA_CODE CBSA_NAME STATE_CODE
## 10214              44201              Ozone        NA                   37
## 10215              44201              Ozone        NA                   37
## 10216              44201              Ozone        NA                   37
## 10217              44201              Ozone        NA                   37
## 10218              44201              Ozone        NA                   37
## 10219              44201              Ozone        NA                   37
##                STATE COUNTY_CODE COUNTY SITE_LATITUDE SITE_LONGITUDE
## 10214 North Carolina         199 Yancey      35.76541      -82.26494
## 10215 North Carolina         199 Yancey      35.76541      -82.26494
## 10216 North Carolina         199 Yancey      35.76541      -82.26494
## 10217 North Carolina         199 Yancey      35.76541      -82.26494
## 10218 North Carolina         199 Yancey      35.76541      -82.26494
## 10219 North Carolina         199 Yancey      35.76541      -82.26494
\end{verbatim}

\begin{Shaded}
\begin{Highlighting}[]
\KeywordTok{summary}\NormalTok{(O3_NC2017}\OperatorTok{$}\NormalTok{DAILY_AQI_VALUE)}
\end{Highlighting}
\end{Shaded}

\begin{verbatim}
##    Min. 1st Qu.  Median    Mean 3rd Qu.    Max. 
##    5.00   32.00   40.00   39.87   45.00  115.00
\end{verbatim}

\begin{Shaded}
\begin{Highlighting}[]
\KeywordTok{class}\NormalTok{(O3_NC2017}\OperatorTok{$}\NormalTok{Date)}
\end{Highlighting}
\end{Shaded}

\begin{verbatim}
## [1] "factor"
\end{verbatim}

\begin{Shaded}
\begin{Highlighting}[]
\CommentTok{# explore O3_NC2018}
\KeywordTok{dim}\NormalTok{(O3_NC2018)}
\end{Highlighting}
\end{Shaded}

\begin{verbatim}
## [1] 10781    20
\end{verbatim}

\begin{Shaded}
\begin{Highlighting}[]
\KeywordTok{colnames}\NormalTok{(O3_NC2018)}
\end{Highlighting}
\end{Shaded}

\begin{verbatim}
##  [1] "Date"                                
##  [2] "Source"                              
##  [3] "Site.ID"                             
##  [4] "POC"                                 
##  [5] "Daily.Max.8.hour.Ozone.Concentration"
##  [6] "UNITS"                               
##  [7] "DAILY_AQI_VALUE"                     
##  [8] "Site.Name"                           
##  [9] "DAILY_OBS_COUNT"                     
## [10] "PERCENT_COMPLETE"                    
## [11] "AQS_PARAMETER_CODE"                  
## [12] "AQS_PARAMETER_DESC"                  
## [13] "CBSA_CODE"                           
## [14] "CBSA_NAME"                           
## [15] "STATE_CODE"                          
## [16] "STATE"                               
## [17] "COUNTY_CODE"                         
## [18] "COUNTY"                              
## [19] "SITE_LATITUDE"                       
## [20] "SITE_LONGITUDE"
\end{verbatim}

\begin{Shaded}
\begin{Highlighting}[]
\KeywordTok{class}\NormalTok{(O3_NC2018)}
\end{Highlighting}
\end{Shaded}

\begin{verbatim}
## [1] "data.frame"
\end{verbatim}

\begin{Shaded}
\begin{Highlighting}[]
\KeywordTok{head}\NormalTok{(O3_NC2018)}
\end{Highlighting}
\end{Shaded}

\begin{verbatim}
##      Date Source   Site.ID POC Daily.Max.8.hour.Ozone.Concentration UNITS
## 1 2/16/18 AirNow 370030005   1                                0.038   ppm
## 2 2/17/18 AirNow 370030005   1                                0.033   ppm
## 3 2/18/18 AirNow 370030005   1                                0.040   ppm
## 4 2/19/18 AirNow 370030005   1                                0.020   ppm
## 5 2/20/18 AirNow 370030005   1                                0.019   ppm
## 6 2/21/18 AirNow 370030005   1                                0.021   ppm
##   DAILY_AQI_VALUE             Site.Name DAILY_OBS_COUNT PERCENT_COMPLETE
## 1              35 Taylorsville Liledoun              24              100
## 2              31 Taylorsville Liledoun              24              100
## 3              37 Taylorsville Liledoun              24              100
## 4              19 Taylorsville Liledoun              24              100
## 5              18 Taylorsville Liledoun              24              100
## 6              19 Taylorsville Liledoun              24              100
##   AQS_PARAMETER_CODE AQS_PARAMETER_DESC CBSA_CODE
## 1              44201              Ozone     25860
## 2              44201              Ozone     25860
## 3              44201              Ozone     25860
## 4              44201              Ozone     25860
## 5              44201              Ozone     25860
## 6              44201              Ozone     25860
##                      CBSA_NAME STATE_CODE          STATE COUNTY_CODE
## 1 Hickory-Lenoir-Morganton, NC         37 North Carolina           3
## 2 Hickory-Lenoir-Morganton, NC         37 North Carolina           3
## 3 Hickory-Lenoir-Morganton, NC         37 North Carolina           3
## 4 Hickory-Lenoir-Morganton, NC         37 North Carolina           3
## 5 Hickory-Lenoir-Morganton, NC         37 North Carolina           3
## 6 Hickory-Lenoir-Morganton, NC         37 North Carolina           3
##      COUNTY SITE_LATITUDE SITE_LONGITUDE
## 1 Alexander       35.9138        -81.191
## 2 Alexander       35.9138        -81.191
## 3 Alexander       35.9138        -81.191
## 4 Alexander       35.9138        -81.191
## 5 Alexander       35.9138        -81.191
## 6 Alexander       35.9138        -81.191
\end{verbatim}

\begin{Shaded}
\begin{Highlighting}[]
\KeywordTok{tail}\NormalTok{(O3_NC2018)}
\end{Highlighting}
\end{Shaded}

\begin{verbatim}
##           Date Source   Site.ID POC Daily.Max.8.hour.Ozone.Concentration
## 10776  11/4/18 AirNow 371990004   1                                0.043
## 10777  11/5/18 AirNow 371990004   1                                0.044
## 10778  11/6/18 AirNow 371990004   1                                0.053
## 10779  11/7/18 AirNow 371990004   1                                0.053
## 10780  11/8/18 AirNow 371990004   1                                0.039
## 10781 11/11/18 AirNow 371990004   1                                0.059
##       UNITS DAILY_AQI_VALUE    Site.Name DAILY_OBS_COUNT PERCENT_COMPLETE
## 10776   ppm              40 Mt. Mitchell              24              100
## 10777   ppm              41 Mt. Mitchell              24              100
## 10778   ppm              49 Mt. Mitchell              24              100
## 10779   ppm              49 Mt. Mitchell              24              100
## 10780   ppm              36 Mt. Mitchell              24              100
## 10781   ppm              64 Mt. Mitchell              24              100
##       AQS_PARAMETER_CODE AQS_PARAMETER_DESC CBSA_CODE CBSA_NAME STATE_CODE
## 10776              44201              Ozone        NA                   37
## 10777              44201              Ozone        NA                   37
## 10778              44201              Ozone        NA                   37
## 10779              44201              Ozone        NA                   37
## 10780              44201              Ozone        NA                   37
## 10781              44201              Ozone        NA                   37
##                STATE COUNTY_CODE COUNTY SITE_LATITUDE SITE_LONGITUDE
## 10776 North Carolina         199 Yancey      35.76541      -82.26494
## 10777 North Carolina         199 Yancey      35.76541      -82.26494
## 10778 North Carolina         199 Yancey      35.76541      -82.26494
## 10779 North Carolina         199 Yancey      35.76541      -82.26494
## 10780 North Carolina         199 Yancey      35.76541      -82.26494
## 10781 North Carolina         199 Yancey      35.76541      -82.26494
\end{verbatim}

\begin{Shaded}
\begin{Highlighting}[]
\KeywordTok{summary}\NormalTok{(O3_NC2018}\OperatorTok{$}\NormalTok{DAILY_AQI_VALUE)}
\end{Highlighting}
\end{Shaded}

\begin{verbatim}
##    Min. 1st Qu.  Median    Mean 3rd Qu.    Max. 
##    0.00   31.00   38.00   39.46   45.00  122.00
\end{verbatim}

\begin{Shaded}
\begin{Highlighting}[]
\KeywordTok{class}\NormalTok{(O3_NC2018}\OperatorTok{$}\NormalTok{Date)}
\end{Highlighting}
\end{Shaded}

\begin{verbatim}
## [1] "factor"
\end{verbatim}

\begin{Shaded}
\begin{Highlighting}[]
\CommentTok{# explore PM25_NC2017}
\KeywordTok{dim}\NormalTok{(PM25_NC2017)}
\end{Highlighting}
\end{Shaded}

\begin{verbatim}
## [1] 9494   20
\end{verbatim}

\begin{Shaded}
\begin{Highlighting}[]
\KeywordTok{colnames}\NormalTok{(PM25_NC2017)}
\end{Highlighting}
\end{Shaded}

\begin{verbatim}
##  [1] "Date"                           "Source"                        
##  [3] "Site.ID"                        "POC"                           
##  [5] "Daily.Mean.PM2.5.Concentration" "UNITS"                         
##  [7] "DAILY_AQI_VALUE"                "Site.Name"                     
##  [9] "DAILY_OBS_COUNT"                "PERCENT_COMPLETE"              
## [11] "AQS_PARAMETER_CODE"             "AQS_PARAMETER_DESC"            
## [13] "CBSA_CODE"                      "CBSA_NAME"                     
## [15] "STATE_CODE"                     "STATE"                         
## [17] "COUNTY_CODE"                    "COUNTY"                        
## [19] "SITE_LATITUDE"                  "SITE_LONGITUDE"
\end{verbatim}

\begin{Shaded}
\begin{Highlighting}[]
\KeywordTok{class}\NormalTok{(PM25_NC2017)}
\end{Highlighting}
\end{Shaded}

\begin{verbatim}
## [1] "data.frame"
\end{verbatim}

\begin{Shaded}
\begin{Highlighting}[]
\KeywordTok{head}\NormalTok{(PM25_NC2017)}
\end{Highlighting}
\end{Shaded}

\begin{verbatim}
##      Date Source   Site.ID POC Daily.Mean.PM2.5.Concentration    UNITS
## 1  1/1/17    AQS 370110002   1                            2.9 ug/m3 LC
## 2  1/4/17    AQS 370110002   1                            1.2 ug/m3 LC
## 3  1/7/17    AQS 370110002   1                            3.2 ug/m3 LC
## 4 1/10/17    AQS 370110002   1                            6.4 ug/m3 LC
## 5 1/13/17    AQS 370110002   1                            3.6 ug/m3 LC
## 6 1/16/17    AQS 370110002   1                            5.8 ug/m3 LC
##   DAILY_AQI_VALUE      Site.Name DAILY_OBS_COUNT PERCENT_COMPLETE
## 1              12 Linville Falls               1              100
## 2               5 Linville Falls               1              100
## 3              13 Linville Falls               1              100
## 4              27 Linville Falls               1              100
## 5              15 Linville Falls               1              100
## 6              24 Linville Falls               1              100
##   AQS_PARAMETER_CODE                     AQS_PARAMETER_DESC CBSA_CODE
## 1              88502 Acceptable PM2.5 AQI & Speciation Mass        NA
## 2              88502 Acceptable PM2.5 AQI & Speciation Mass        NA
## 3              88502 Acceptable PM2.5 AQI & Speciation Mass        NA
## 4              88502 Acceptable PM2.5 AQI & Speciation Mass        NA
## 5              88502 Acceptable PM2.5 AQI & Speciation Mass        NA
## 6              88502 Acceptable PM2.5 AQI & Speciation Mass        NA
##   CBSA_NAME STATE_CODE          STATE COUNTY_CODE COUNTY SITE_LATITUDE
## 1                   37 North Carolina          11  Avery      35.97235
## 2                   37 North Carolina          11  Avery      35.97235
## 3                   37 North Carolina          11  Avery      35.97235
## 4                   37 North Carolina          11  Avery      35.97235
## 5                   37 North Carolina          11  Avery      35.97235
## 6                   37 North Carolina          11  Avery      35.97235
##   SITE_LONGITUDE
## 1      -81.93307
## 2      -81.93307
## 3      -81.93307
## 4      -81.93307
## 5      -81.93307
## 6      -81.93307
\end{verbatim}

\begin{Shaded}
\begin{Highlighting}[]
\KeywordTok{tail}\NormalTok{(PM25_NC2017)}
\end{Highlighting}
\end{Shaded}

\begin{verbatim}
##          Date Source   Site.ID POC Daily.Mean.PM2.5.Concentration    UNITS
## 9489 12/26/17    AQS 371830021   3                            4.1 ug/m3 LC
## 9490 12/27/17    AQS 371830021   3                            7.2 ug/m3 LC
## 9491 12/28/17    AQS 371830021   3                            7.1 ug/m3 LC
## 9492 12/29/17    AQS 371830021   3                           11.6 ug/m3 LC
## 9493 12/30/17    AQS 371830021   3                           15.3 ug/m3 LC
## 9494 12/31/17    AQS 371830021   3                            2.9 ug/m3 LC
##      DAILY_AQI_VALUE  Site.Name DAILY_OBS_COUNT PERCENT_COMPLETE
## 9489              17 Triple Oak               1              100
## 9490              30 Triple Oak               1              100
## 9491              30 Triple Oak               1              100
## 9492              48 Triple Oak               1              100
## 9493              58 Triple Oak               1              100
## 9494              12 Triple Oak               1              100
##      AQS_PARAMETER_CODE       AQS_PARAMETER_DESC CBSA_CODE   CBSA_NAME
## 9489              88101 PM2.5 - Local Conditions     39580 Raleigh, NC
## 9490              88101 PM2.5 - Local Conditions     39580 Raleigh, NC
## 9491              88101 PM2.5 - Local Conditions     39580 Raleigh, NC
## 9492              88101 PM2.5 - Local Conditions     39580 Raleigh, NC
## 9493              88101 PM2.5 - Local Conditions     39580 Raleigh, NC
## 9494              88101 PM2.5 - Local Conditions     39580 Raleigh, NC
##      STATE_CODE          STATE COUNTY_CODE COUNTY SITE_LATITUDE
## 9489         37 North Carolina         183   Wake       35.8652
## 9490         37 North Carolina         183   Wake       35.8652
## 9491         37 North Carolina         183   Wake       35.8652
## 9492         37 North Carolina         183   Wake       35.8652
## 9493         37 North Carolina         183   Wake       35.8652
## 9494         37 North Carolina         183   Wake       35.8652
##      SITE_LONGITUDE
## 9489       -78.8197
## 9490       -78.8197
## 9491       -78.8197
## 9492       -78.8197
## 9493       -78.8197
## 9494       -78.8197
\end{verbatim}

\begin{Shaded}
\begin{Highlighting}[]
\KeywordTok{summary}\NormalTok{(PM25_NC2017}\OperatorTok{$}\NormalTok{DAILY_AQI_VALUE)}
\end{Highlighting}
\end{Shaded}

\begin{verbatim}
##    Min. 1st Qu.  Median    Mean 3rd Qu.    Max. 
##    0.00   21.00   30.00   31.72   42.00   93.00
\end{verbatim}

\begin{Shaded}
\begin{Highlighting}[]
\KeywordTok{class}\NormalTok{(PM25_NC2017}\OperatorTok{$}\NormalTok{Date)}
\end{Highlighting}
\end{Shaded}

\begin{verbatim}
## [1] "factor"
\end{verbatim}

\begin{Shaded}
\begin{Highlighting}[]
\CommentTok{# explore PM25_NC2018}
\KeywordTok{dim}\NormalTok{(PM25_NC2018)}
\end{Highlighting}
\end{Shaded}

\begin{verbatim}
## [1] 7611   20
\end{verbatim}

\begin{Shaded}
\begin{Highlighting}[]
\KeywordTok{colnames}\NormalTok{(PM25_NC2018)}
\end{Highlighting}
\end{Shaded}

\begin{verbatim}
##  [1] "Date"                           "Source"                        
##  [3] "Site.ID"                        "POC"                           
##  [5] "Daily.Mean.PM2.5.Concentration" "UNITS"                         
##  [7] "DAILY_AQI_VALUE"                "Site.Name"                     
##  [9] "DAILY_OBS_COUNT"                "PERCENT_COMPLETE"              
## [11] "AQS_PARAMETER_CODE"             "AQS_PARAMETER_DESC"            
## [13] "CBSA_CODE"                      "CBSA_NAME"                     
## [15] "STATE_CODE"                     "STATE"                         
## [17] "COUNTY_CODE"                    "COUNTY"                        
## [19] "SITE_LATITUDE"                  "SITE_LONGITUDE"
\end{verbatim}

\begin{Shaded}
\begin{Highlighting}[]
\KeywordTok{class}\NormalTok{(PM25_NC2018)}
\end{Highlighting}
\end{Shaded}

\begin{verbatim}
## [1] "data.frame"
\end{verbatim}

\begin{Shaded}
\begin{Highlighting}[]
\KeywordTok{head}\NormalTok{(PM25_NC2018)}
\end{Highlighting}
\end{Shaded}

\begin{verbatim}
##      Date Source   Site.ID POC Daily.Mean.PM2.5.Concentration    UNITS
## 1  1/2/18    AQS 370110002   1                            2.9 ug/m3 LC
## 2  1/5/18    AQS 370110002   1                            3.7 ug/m3 LC
## 3  1/8/18    AQS 370110002   1                            5.3 ug/m3 LC
## 4 1/11/18    AQS 370110002   1                            0.8 ug/m3 LC
## 5 1/14/18    AQS 370110002   1                            2.5 ug/m3 LC
## 6 1/17/18    AQS 370110002   1                            4.5 ug/m3 LC
##   DAILY_AQI_VALUE      Site.Name DAILY_OBS_COUNT PERCENT_COMPLETE
## 1              12 Linville Falls               1              100
## 2              15 Linville Falls               1              100
## 3              22 Linville Falls               1              100
## 4               3 Linville Falls               1              100
## 5              10 Linville Falls               1              100
## 6              19 Linville Falls               1              100
##   AQS_PARAMETER_CODE                     AQS_PARAMETER_DESC CBSA_CODE
## 1              88502 Acceptable PM2.5 AQI & Speciation Mass        NA
## 2              88502 Acceptable PM2.5 AQI & Speciation Mass        NA
## 3              88502 Acceptable PM2.5 AQI & Speciation Mass        NA
## 4              88502 Acceptable PM2.5 AQI & Speciation Mass        NA
## 5              88502 Acceptable PM2.5 AQI & Speciation Mass        NA
## 6              88502 Acceptable PM2.5 AQI & Speciation Mass        NA
##   CBSA_NAME STATE_CODE          STATE COUNTY_CODE COUNTY SITE_LATITUDE
## 1                   37 North Carolina          11  Avery      35.97235
## 2                   37 North Carolina          11  Avery      35.97235
## 3                   37 North Carolina          11  Avery      35.97235
## 4                   37 North Carolina          11  Avery      35.97235
## 5                   37 North Carolina          11  Avery      35.97235
## 6                   37 North Carolina          11  Avery      35.97235
##   SITE_LONGITUDE
## 1      -81.93307
## 2      -81.93307
## 3      -81.93307
## 4      -81.93307
## 5      -81.93307
## 6      -81.93307
\end{verbatim}

\begin{Shaded}
\begin{Highlighting}[]
\KeywordTok{tail}\NormalTok{(O3_NC2018)}
\end{Highlighting}
\end{Shaded}

\begin{verbatim}
##           Date Source   Site.ID POC Daily.Max.8.hour.Ozone.Concentration
## 10776  11/4/18 AirNow 371990004   1                                0.043
## 10777  11/5/18 AirNow 371990004   1                                0.044
## 10778  11/6/18 AirNow 371990004   1                                0.053
## 10779  11/7/18 AirNow 371990004   1                                0.053
## 10780  11/8/18 AirNow 371990004   1                                0.039
## 10781 11/11/18 AirNow 371990004   1                                0.059
##       UNITS DAILY_AQI_VALUE    Site.Name DAILY_OBS_COUNT PERCENT_COMPLETE
## 10776   ppm              40 Mt. Mitchell              24              100
## 10777   ppm              41 Mt. Mitchell              24              100
## 10778   ppm              49 Mt. Mitchell              24              100
## 10779   ppm              49 Mt. Mitchell              24              100
## 10780   ppm              36 Mt. Mitchell              24              100
## 10781   ppm              64 Mt. Mitchell              24              100
##       AQS_PARAMETER_CODE AQS_PARAMETER_DESC CBSA_CODE CBSA_NAME STATE_CODE
## 10776              44201              Ozone        NA                   37
## 10777              44201              Ozone        NA                   37
## 10778              44201              Ozone        NA                   37
## 10779              44201              Ozone        NA                   37
## 10780              44201              Ozone        NA                   37
## 10781              44201              Ozone        NA                   37
##                STATE COUNTY_CODE COUNTY SITE_LATITUDE SITE_LONGITUDE
## 10776 North Carolina         199 Yancey      35.76541      -82.26494
## 10777 North Carolina         199 Yancey      35.76541      -82.26494
## 10778 North Carolina         199 Yancey      35.76541      -82.26494
## 10779 North Carolina         199 Yancey      35.76541      -82.26494
## 10780 North Carolina         199 Yancey      35.76541      -82.26494
## 10781 North Carolina         199 Yancey      35.76541      -82.26494
\end{verbatim}

\begin{Shaded}
\begin{Highlighting}[]
\KeywordTok{summary}\NormalTok{(PM25_NC2018}\OperatorTok{$}\NormalTok{DAILY_AQI_VALUE)}
\end{Highlighting}
\end{Shaded}

\begin{verbatim}
##    Min. 1st Qu.  Median    Mean 3rd Qu.    Max. 
##    0.00   21.00   30.00   31.03   41.00   97.00
\end{verbatim}

\begin{Shaded}
\begin{Highlighting}[]
\KeywordTok{class}\NormalTok{(PM25_NC2018}\OperatorTok{$}\NormalTok{Date)}
\end{Highlighting}
\end{Shaded}

\begin{verbatim}
## [1] "factor"
\end{verbatim}

\subsection{Wrangle individual datasets to create processed
files.}\label{wrangle-individual-datasets-to-create-processed-files.}

\begin{enumerate}
\def\labelenumi{\arabic{enumi}.}
\setcounter{enumi}{2}
\tightlist
\item
  Change date to date
\item
  Select the following columns: Date, DAILY\_AQI\_VALUE, Site.Name,
  AQS\_PARAMETER\_DESC, COUNTY, SITE\_LATITUDE, SITE\_LONGITUDE
\item
  For the PM2.5 datasets, fill all cells in AQS\_PARAMETER\_DESC with
  ``PM2.5'' (all cells in this column should be identical).
\item
  Save all four processed datasets in the Processed folder.
\end{enumerate}

\begin{Shaded}
\begin{Highlighting}[]
\CommentTok{#3}

\CommentTok{# check class of column date}
\KeywordTok{class}\NormalTok{(O3_NC2017}\OperatorTok{$}\NormalTok{Date)}
\end{Highlighting}
\end{Shaded}

\begin{verbatim}
## [1] "factor"
\end{verbatim}

\begin{Shaded}
\begin{Highlighting}[]
\KeywordTok{class}\NormalTok{(O3_NC2018}\OperatorTok{$}\NormalTok{Date)}
\end{Highlighting}
\end{Shaded}

\begin{verbatim}
## [1] "factor"
\end{verbatim}

\begin{Shaded}
\begin{Highlighting}[]
\KeywordTok{class}\NormalTok{(PM25_NC2017}\OperatorTok{$}\NormalTok{Date)}
\end{Highlighting}
\end{Shaded}

\begin{verbatim}
## [1] "factor"
\end{verbatim}

\begin{Shaded}
\begin{Highlighting}[]
\KeywordTok{class}\NormalTok{(PM25_NC2018}\OperatorTok{$}\NormalTok{Date)}
\end{Highlighting}
\end{Shaded}

\begin{verbatim}
## [1] "factor"
\end{verbatim}

\begin{Shaded}
\begin{Highlighting}[]
\CommentTok{# change the date columns from class factor to date}
\NormalTok{O3_NC2017}\OperatorTok{$}\NormalTok{Date <-}\StringTok{ }\KeywordTok{as.Date}\NormalTok{(O3_NC2017}\OperatorTok{$}\NormalTok{Date, }\DataTypeTok{format =} \StringTok{"%m/%d/%y"}\NormalTok{)}
\NormalTok{O3_NC2018}\OperatorTok{$}\NormalTok{Date <-}\StringTok{ }\KeywordTok{as.Date}\NormalTok{(O3_NC2018}\OperatorTok{$}\NormalTok{Date, }\DataTypeTok{format =} \StringTok{"%m/%d/%y"}\NormalTok{)}
\NormalTok{PM25_NC2017}\OperatorTok{$}\NormalTok{Date <-}\StringTok{ }\KeywordTok{as.Date}\NormalTok{(PM25_NC2017}\OperatorTok{$}\NormalTok{Date, }\DataTypeTok{format =} \StringTok{"%m/%d/%y"}\NormalTok{)}
\NormalTok{PM25_NC2018}\OperatorTok{$}\NormalTok{Date <-}\StringTok{ }\KeywordTok{as.Date}\NormalTok{(PM25_NC2018}\OperatorTok{$}\NormalTok{Date, }\DataTypeTok{format =} \StringTok{"%m/%d/%y"}\NormalTok{)}

\CommentTok{# confirm date columns are class date}
\KeywordTok{class}\NormalTok{(O3_NC2017}\OperatorTok{$}\NormalTok{Date)}
\end{Highlighting}
\end{Shaded}

\begin{verbatim}
## [1] "Date"
\end{verbatim}

\begin{Shaded}
\begin{Highlighting}[]
\KeywordTok{class}\NormalTok{(O3_NC2018}\OperatorTok{$}\NormalTok{Date)}
\end{Highlighting}
\end{Shaded}

\begin{verbatim}
## [1] "Date"
\end{verbatim}

\begin{Shaded}
\begin{Highlighting}[]
\KeywordTok{class}\NormalTok{(PM25_NC2017}\OperatorTok{$}\NormalTok{Date)}
\end{Highlighting}
\end{Shaded}

\begin{verbatim}
## [1] "Date"
\end{verbatim}

\begin{Shaded}
\begin{Highlighting}[]
\KeywordTok{class}\NormalTok{(PM25_NC2018}\OperatorTok{$}\NormalTok{Date)}
\end{Highlighting}
\end{Shaded}

\begin{verbatim}
## [1] "Date"
\end{verbatim}

\begin{Shaded}
\begin{Highlighting}[]
\CommentTok{#4}

\CommentTok{# create processed datasets}
\NormalTok{O3_NC2017_processed <-}\StringTok{ }\KeywordTok{select}\NormalTok{(O3_NC2017, Date, DAILY_AQI_VALUE, Site.Name, }
\NormalTok{                              AQS_PARAMETER_DESC, COUNTY, SITE_LATITUDE, SITE_LONGITUDE)}
\NormalTok{O3_NC2018_processed <-}\StringTok{ }\KeywordTok{select}\NormalTok{(O3_NC2018, Date, DAILY_AQI_VALUE, Site.Name, }
\NormalTok{                              AQS_PARAMETER_DESC, COUNTY, SITE_LATITUDE, SITE_LONGITUDE)}
\NormalTok{PM25_NC2017_processed <-}\StringTok{ }\KeywordTok{select}\NormalTok{(PM25_NC2017, Date, DAILY_AQI_VALUE, Site.Name, }
\NormalTok{                                AQS_PARAMETER_DESC, COUNTY, SITE_LATITUDE, SITE_LONGITUDE)}
\NormalTok{PM25_NC2018_processed <-}\StringTok{ }\KeywordTok{select}\NormalTok{(PM25_NC2018, Date, DAILY_AQI_VALUE, Site.Name, }
\NormalTok{                                AQS_PARAMETER_DESC, COUNTY, SITE_LATITUDE, SITE_LONGITUDE)}

\CommentTok{#5}

\CommentTok{# For the PM2.5 datasets, fill all cells in AQS_PARAMETER_DESC with "PM2.5"}

\NormalTok{PM25_NC2017_processed <-}\StringTok{ }\KeywordTok{mutate}\NormalTok{(PM25_NC2017_processed, }\DataTypeTok{AQS_PARAMETER_DESC =} \StringTok{"PM2.5"}\NormalTok{)}

\NormalTok{PM25_NC2018_processed <-}\StringTok{ }\KeywordTok{mutate}\NormalTok{(PM25_NC2018_processed, }\DataTypeTok{AQS_PARAMETER_DESC =} \StringTok{"PM2.5"}\NormalTok{)}

\KeywordTok{class}\NormalTok{(PM25_NC2017_processed}\OperatorTok{$}\NormalTok{AQS_PARAMETER_DESC)}
\end{Highlighting}
\end{Shaded}

\begin{verbatim}
## [1] "character"
\end{verbatim}

\begin{Shaded}
\begin{Highlighting}[]
\CommentTok{#note: the column AQS_PARAMETER_DESC has changed from class factor to character}

\CommentTok{#6}

\CommentTok{# save processed datasets in processed folder}

\KeywordTok{write.csv}\NormalTok{(O3_NC2017_processed, }\DataTypeTok{row.names =} \OtherTok{FALSE}\NormalTok{, }
          \DataTypeTok{file =} \StringTok{"./Data/Processed/EPAair_O3_NC2017_processed.csv"}\NormalTok{)}
\KeywordTok{write.csv}\NormalTok{(O3_NC2018_processed, }\DataTypeTok{row.names =} \OtherTok{FALSE}\NormalTok{, }
          \DataTypeTok{file =} \StringTok{"./Data/Processed/EPAair_O3_NC2018_processed.csv"}\NormalTok{)}
\KeywordTok{write.csv}\NormalTok{(PM25_NC2017_processed, }\DataTypeTok{row.names =} \OtherTok{FALSE}\NormalTok{, }
          \DataTypeTok{file =} \StringTok{"./Data/Processed/EPAair_PM25_NC2017_processed.csv"}\NormalTok{)}
\KeywordTok{write.csv}\NormalTok{(PM25_NC2018_processed, }\DataTypeTok{row.names =} \OtherTok{FALSE}\NormalTok{, }
          \DataTypeTok{file =} \StringTok{"./Data/Processed/EPAair_PM25_NC2018_processed.csv"}\NormalTok{)}
\end{Highlighting}
\end{Shaded}

\subsection{Combine datasets}\label{combine-datasets}

\begin{enumerate}
\def\labelenumi{\arabic{enumi}.}
\setcounter{enumi}{6}
\tightlist
\item
  Combine the four datasets with \texttt{rbind}. Make sure your column
  names are identical prior to running this code.
\item
  Wrangle your new dataset with a pipe function (\%\textgreater{}\%) so
  that it fills the following conditions:
\end{enumerate}

\begin{itemize}
\tightlist
\item
  Sites: Blackstone, Bryson City, Triple Oak
\item
  Add columns for ``Month'' and ``Year'' by parsing your ``Date'' column
  (hint: \texttt{separate} function or \texttt{lubridate} package)
\end{itemize}

\begin{enumerate}
\def\labelenumi{\arabic{enumi}.}
\setcounter{enumi}{8}
\tightlist
\item
  Spread your datasets such that AQI values for ozone and PM2.5 are in
  separate columns. Each location on a specific date should now occupy
  only one row.
\item
  Call up the dimensions of your new tidy dataset.
\item
  Save your processed dataset with the following file name:
  ``EPAair\_O3\_PM25\_NC1718\_Processed.csv''
\end{enumerate}

\begin{Shaded}
\begin{Highlighting}[]
\CommentTok{#7}

\CommentTok{# check for identical column names across the 4 datasets}
\KeywordTok{all}\NormalTok{(}\KeywordTok{colnames}\NormalTok{(O3_NC2017_processed) }\OperatorTok{==}\StringTok{ }\KeywordTok{colnames}\NormalTok{(O3_NC2018_processed))}
\end{Highlighting}
\end{Shaded}

\begin{verbatim}
## [1] TRUE
\end{verbatim}

\begin{Shaded}
\begin{Highlighting}[]
\KeywordTok{all}\NormalTok{(}\KeywordTok{colnames}\NormalTok{(PM25_NC2017_processed) }\OperatorTok{==}\StringTok{ }\KeywordTok{colnames}\NormalTok{(PM25_NC2018_processed))}
\end{Highlighting}
\end{Shaded}

\begin{verbatim}
## [1] TRUE
\end{verbatim}

\begin{Shaded}
\begin{Highlighting}[]
\KeywordTok{all}\NormalTok{(}\KeywordTok{colnames}\NormalTok{(O3_NC2017_processed) }\OperatorTok{==}\StringTok{ }\KeywordTok{colnames}\NormalTok{(PM25_NC2017_processed))}
\end{Highlighting}
\end{Shaded}

\begin{verbatim}
## [1] TRUE
\end{verbatim}

\begin{Shaded}
\begin{Highlighting}[]
\CommentTok{# combine datasets vertically}
\NormalTok{EPAair_O3_PM25 <-}\StringTok{ }\KeywordTok{rbind}\NormalTok{(O3_NC2017_processed, O3_NC2018_processed, }
\NormalTok{                        PM25_NC2017_processed, PM25_NC2018_processed)}

\CommentTok{#8}

\CommentTok{# wrangle based on conditions}
\NormalTok{EPAair_O3_PM25_processed <-}
\NormalTok{EPAair_O3_PM25 }\OperatorTok
\KeywordTok{filter}\NormalTok{(Site.Name }\OperatorTok{==}\StringTok{ "Blackstone"} \OperatorTok{|}\StringTok{ }\NormalTok{Site.Name }\OperatorTok{==}\StringTok{ "Bryson City"} \OperatorTok{|}\StringTok{ }\NormalTok{Site.Name }\OperatorTok{==}\StringTok{ "Triple Oak"}\NormalTok{) }\OperatorTok
\KeywordTok{separate}\NormalTok{(Date, }\KeywordTok{c}\NormalTok{(}\StringTok{"Y"}\NormalTok{, }\StringTok{"m"}\NormalTok{, }\StringTok{"d"}\NormalTok{)) }\OperatorTok
\KeywordTok{rename}\NormalTok{(}\DataTypeTok{Year =} \StringTok{"Y"}\NormalTok{) }\OperatorTok
\KeywordTok{rename}\NormalTok{(}\DataTypeTok{Month =} \StringTok{"m"}\NormalTok{) }\OperatorTok
\KeywordTok{rename}\NormalTok{(}\DataTypeTok{Day =} \StringTok{"d"}\NormalTok{)}

\CommentTok{#9}

\CommentTok{# spread dataset so ozone and PM2.5 are in separate columns}
\NormalTok{EPAair_O3_PM25_spread <-}\StringTok{ }\KeywordTok{spread}\NormalTok{(EPAair_O3_PM25_processed, AQS_PARAMETER_DESC, DAILY_AQI_VALUE)}

\CommentTok{#10}

\CommentTok{# check dimensions of spread dataset}
\KeywordTok{dim}\NormalTok{(EPAair_O3_PM25_spread)}
\end{Highlighting}
\end{Shaded}

\begin{verbatim}
## [1] 1953    9
\end{verbatim}

\begin{Shaded}
\begin{Highlighting}[]
\CommentTok{#11}

\CommentTok{# save spread dataset}
\KeywordTok{write.csv}\NormalTok{(EPAair_O3_PM25_spread, }\DataTypeTok{row.names =} \OtherTok{FALSE}\NormalTok{, }
          \DataTypeTok{file =} \StringTok{"./Data/Processed/EPAair_O3_PM25_NC1718_Processed.csv"}\NormalTok{)}
\end{Highlighting}
\end{Shaded}

\subsection{Generate summary tables}\label{generate-summary-tables}

\begin{enumerate}
\def\labelenumi{\arabic{enumi}.}
\setcounter{enumi}{11}
\tightlist
\item
  Use the split-apply-combine strategy to generate two new data frames:
\end{enumerate}

\begin{enumerate}
\def\labelenumi{\alph{enumi}.}
\tightlist
\item
  A summary table of mean AQI values for O3 and PM2.5 by month
\item
  A summary table of the mean, minimum, and maximum AQI values of O3 and
  PM2.5 for each site
\end{enumerate}

\begin{enumerate}
\def\labelenumi{\arabic{enumi}.}
\setcounter{enumi}{12}
\tightlist
\item
  Display the data frames.
\end{enumerate}

\begin{Shaded}
\begin{Highlighting}[]
\CommentTok{#12a}

\CommentTok{# create summary of mean O3 and PM2.5 by month}
\NormalTok{mean_O3_PM25_bymonth <-}
\StringTok{  }\NormalTok{EPAair_O3_PM25_spread }\OperatorTok
\StringTok{  }\KeywordTok{group_by}\NormalTok{(Month) }\OperatorTok
\StringTok{  }\KeywordTok{summarise}\NormalTok{(}\DataTypeTok{meanO3 =} \KeywordTok{mean}\NormalTok{(Ozone, }\DataTypeTok{na.rm=}\OtherTok{TRUE}\NormalTok{),}
            \DataTypeTok{meanPM25 =} \KeywordTok{mean}\NormalTok{(PM2.}\DecValTok{5}\NormalTok{, }\DataTypeTok{na.rm=}\OtherTok{TRUE}\NormalTok{))}

\CommentTok{#12b}

\CommentTok{# create a summary of mean/min/max O3 and PM2.5 by site}
\NormalTok{stats_O3_PM25_bysite <-}
\StringTok{  }\NormalTok{EPAair_O3_PM25_spread }\OperatorTok
\StringTok{  }\KeywordTok{group_by}\NormalTok{(Site.Name) }\OperatorTok
\StringTok{  }\KeywordTok{summarise}\NormalTok{(}\DataTypeTok{meanO3 =} \KeywordTok{mean}\NormalTok{(Ozone, }\DataTypeTok{na.rm=}\OtherTok{TRUE}\NormalTok{),}
            \DataTypeTok{meanPM25 =} \KeywordTok{mean}\NormalTok{(PM2.}\DecValTok{5}\NormalTok{, }\DataTypeTok{na.rm=}\OtherTok{TRUE}\NormalTok{),}
            \DataTypeTok{minO3 =} \KeywordTok{min}\NormalTok{(Ozone, }\DataTypeTok{na.rm=}\OtherTok{TRUE}\NormalTok{),}
            \DataTypeTok{minPM25 =} \KeywordTok{min}\NormalTok{(PM2.}\DecValTok{5}\NormalTok{, }\DataTypeTok{na.rm=}\OtherTok{TRUE}\NormalTok{),}
            \DataTypeTok{maxO3 =} \KeywordTok{max}\NormalTok{(Ozone, }\DataTypeTok{na.rm=}\OtherTok{TRUE}\NormalTok{),}
            \DataTypeTok{maxPM25 =} \KeywordTok{max}\NormalTok{(PM2.}\DecValTok{5}\NormalTok{, }\DataTypeTok{na.rm=}\OtherTok{TRUE}\NormalTok{))}

\CommentTok{#13}

\CommentTok{# display dataframes in console}
\KeywordTok{print}\NormalTok{(mean_O3_PM25_bymonth)}
\end{Highlighting}
\end{Shaded}

\begin{verbatim}
## # A tibble: 12 x 3
##    Month meanO3 meanPM25
##    <chr>  <dbl>    <dbl>
##  1 01      31.5     34.6
##  2 02      35.5     36.7
##  3 03      42.4     35.1
##  4 04      44.3     32.5
##  5 05      38.9     31.7
##  6 06      38.7     33.3
##  7 07      38.2     33.1
##  8 08      34.0     33.7
##  9 09      32.6     31.9
## 10 10      32.1     29.3
## 11 11      30.1     36.8
## 12 12      29.8     41.1
\end{verbatim}

\begin{Shaded}
\begin{Highlighting}[]
\KeywordTok{print}\NormalTok{(stats_O3_PM25_bysite)}
\end{Highlighting}
\end{Shaded}

\begin{verbatim}
## # A tibble: 3 x 7
##   Site.Name   meanO3 meanPM25 minO3 minPM25 maxO3 maxPM25
##   <fct>        <dbl>    <dbl> <dbl>   <dbl> <dbl>   <dbl>
## 1 Blackstone    38.5     36.7     8       0    97      83
## 2 Bryson City   35.2     32.3     5       3    71      78
## 3 Triple Oak   NaN       33.5   Inf       0  -Inf      74
\end{verbatim}

\begin{Shaded}
\begin{Highlighting}[]
\CommentTok{# note that the column Month is now a character}
\end{Highlighting}
\end{Shaded}


\end{document}
